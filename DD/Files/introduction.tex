\subsection{Purpose}
The purpose of this document is to provide a more detailed description than the one already given in the RASD document about the SafeStreets system. This document is mainly addressed to developers, as it focuses on the illustration of specific components and design choices. The aim is to guide developers during implementation, integration and testing.\par
The aspects covered by the design document are:
\begin{itemize}
	\item High level architecture of SafeStreets.
	\item Structure of the main components and their respective interfaces through which they communicate.
	\item Expected runtime behaviour.
	\item How requirements defined in the RASD map to the design elements of this document.
	\item Implementation, Integration and Testing plan.
\end{itemize}

\subsection{Scope}
SafeStreets is a software that aims to collect data coming from registered users in the form of reports and allow either basic users, policemen and municipalities to mine this data with different levels of detail.
The stakeholders of the system will be:
\begin{itemize}
	\item Regular Users: Simple users that want to report traffic violations or have insight of the most unsafe areas of a city.
	\item Policemen: Members of the local police that need to access the data about users' reports and possibly take actions based on these. 
	\item Municipal Authorities: Members of the municipality that need to analyse data in order to strengthen city safety.
\end{itemize}
Data coming from the users is collected by SafeStreets servers that organise reports and build statistics out of them. These data allows the system to offer the basic application feature: providing violations information to users, visualized mainly in map format.\\
The system can also collect data coming from municipalities subscribed to SS advanced function and cross them with informations collected from the users. In this way we can enable advanced analysis  features, such as intervention suggestion on a city metropolitan aerea.\\
Finally SS can offer to municipalities  which request it a traffic ticket emissions tracking service, providing statistics and acces also on this kind of data.\\
SS sytems manage requests from different users and save and protect a relevant ammount of data.\newline
A registration is needed to exploit the various functionalities, even for Regular Users, reducing the possibility of false reports. Personal and sensitive informations required during registration process, are quite limited and only collected in order to reduce service abuse. Thus the system ensures an adeguate level of user privacy, reducing at the same time the costs of sensitive data management and protection.\newline
In addiction, every data treatment will be carried on respecting every applicable privacy regulation, such as GDPR, and users will be informed about data collection and its porpuses in a dedicated privacy policy.\newline
To ensure an even higher level of transparency, SafeStreets will ask users permission to access device data or modules, such as GPS and camera, according to last mobile OSes privacy policies.

\subsection{Definitions, Acronyms, Abbreviations, Conventions}
\subsubsection{Definitions}
\begin{itemize}
	\item Smartphone: Any mobile phone that has software like the one running on a small computer, and that connects to the internet.
	\item Computer: Any machine or device that performs processes, calculations and operations based on instructions provided by a software or hardware program.
\end{itemize}
\subsubsection{Acronyms}
\begin{itemize}
	\item RASD: Requirements Analysis and Specification Document
	\item DD: Design Document
	\item API: Application Programming Interface
	\item GPS: Global Positioning System
	\item MVC: Model View Controller
	\item GUI: Graphical User Interface
	\item DB: DataBase
	\item DBMS: DataBase Management System
	\item SS: SafeStreets
	\item HTTP: Hyper Text Transfer Protocol
	\item HTTPS: HTTP over Secure Socket Layer
	\item GDPR: General Data Protection Regulation
	\item OS: Operative System
\end{itemize}
\subsubsection{Abbreviations}
\begin{itemize}
	\item (Gn) : n-Goal
	\item (G1.n): n-Goal for advanced function 1
	\item (G2.n): n-Goal for advanced function 2
	\item (Rn): n-Requirement
\end{itemize}
\subsection{Revision History}
\begin{itemize}
	\item Version 1.0 - 9/12/2019
\end{itemize}
\subsection{Reference documents}
\begin{itemize}
	\item Slides about DD from the course "Software Engineering 2"
	\item Davide Perugini, Antonio Pipita, Stefano Panzeri, \textit{SafeStreets Requirement Analisys and Specification Document}
\end{itemize}
\subsection{Document Structure}
\begin{itemize}
	\item Introduction: this chapter contains the purpose and the scope of the design document. In order to make the document more comprehensible, this chapter contains also an explanation of terms, acronyms and conventions used.
	\item Architectural Design: this section describes in general the architecture of the system including the three most important views: component, deployment and runtime. In this chapter is also described how the various components of the system interact and the reason behind particular design choices.
	\item User Interface Design: this chapter presents a reference to the mock-ups previously presented in the RASD document.
	\item Requirements traceability: this section shows how the components introduced in this document map the requirements given in the RASD.
	\item Implementation, integration and test plan: describes how to plan the implementation and integration of the various components and how to validate the system following the requirements.
	\item Effort spent: shows the number of hours each member of the group spent for every chapter of the document.
\end{itemize}