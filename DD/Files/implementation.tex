In this Section SafeStreets implementation, integration and testing will be discussed.
\\

\subsection{Implementation Plan}
SafeStreets implementation wil start from the data tier moving towards presentation tier. The decision of this development order aims to semplify  the unit-test of every component and to guarantee testing for all component implemented at a given time. This implementation plan follows the flow of information that derives from the proposed client-server structure.\newline
Thus, SafeStreets implementation will follow this order:
\begin{itemize}
	\item Database server: the database server will be the first part to be implemented, since all of its functions (ranging from submitting reports to even allow users to log in and out of the app) depend on it. A relational database management system has been chosen
	\item Database server - application server connection
	\item Application server: the application server components will be implemented in order of relevance, while still paying attention to any relation between the components. Thus, the implementation order will be the following:
	\begin{itemize}
		\item UserRegistration
		\item LogOperation
		\item ReportHandler
		\item ReportRetriever
		\item PoliceActionHandler
		\item DataFilter
		\item QueryOnAnalyzedData
	\end{itemize}
	This sequence is necessary because, in order to be able to perform any kind of action, there must be registered users (UserRegistration) and said users need to be able to log in their accounts(LogOperations) and submit reports(ReportHandler). Also, police has to be able to take action on said reports (ReportRetriever, then PoliceActionHandler) and municipal authorities must be able to mine data and cross database informations (ReportRetriver, then DataFilter). Last but not least, users need to query data mined by the authorities (QueryOnAnalizedData).
	\item clients - application server connection
	\item clients: in order for the application to work in its basic functions RegularUser client will be implemented first, Policemen and MunicipalAuthority clients will follow.\newline
	Order of implementation:
		\begin{itemize}
			\item RegularUser client:
			\begin{itemize}
				\item RegistrationManager
				\item LoginManager
				\item ReportManager
				\item PlateRecognizer
				\item MapViewer
			\end{itemize}
			\newpage
			\item Policeman client:
			\begin{itemize}
				\item PoliceActionManager 
				\item ReportManager (almost equal to the RegularUser client one, but it packs ViolationReport with "dispactchedOfficer" field already compiled)
			\end{itemize}
			Please note that LoginManager, PlateRecogniser and MapViewer have already been implemented
			\item MunicipalAuthoritiesClient:
				\begin{itemize}
					\item SubscriptionManager
					\item RequestManager
					\item CrashDataSender
					\item MiningService
				\end{itemize}
			Please note that LoginManager and MapViewer have already been implemented 
			\end{itemize}
	\item Load balancer: off-the-shelf CISCO solution (for reference, not a mandatory solution)
	\end{itemize} 

\subsection{Integration Plan}
\subsubsection{Basic conditions}
Integration is the process through which the various implemented components are connected and start interacting with each other.
In order to achieve a meaningful integration, there are some basic condition that must be fulfilled. \par
First of all, the most important condition is that every single component must be working correctly and their functionalities must be tested. In this case, if during integration a problem arises, this must be related only to the integration phase. \par
Another important requirement is that is not necessary for every single component's functionality to be implemented, but at least the main ones and those needed to interact with other components must be tested and correctly operating.

\subsubsection{Integration: Database - Application server}
Since Database and Application Server will be the first parts to be implemented, integration between them is of primary importance to ensure the proper conduct of the following operations.
The specific integrations are:
\begin{itemize}
	\item UserRegistration - Database
	\item LogOperation - Database
	\item ReportHandler - Database
	\item ReportRertiever - Database
	\item PoliceActionHandler - Database
	\item DataFilter - Database
	\item QueryOnAnalyzedData - Database
\end{itemize}
\subsubsection{Integration within the Application Server}
The integration within the application server will be as follows:
\begin{itemize}
	\item Router - LogOperations
	\item Router - ReportHandler
	\item Router - QueryOnAnalyzedData
	\item Router - UserRegistration
	\item Router - DataFilter
	\item Router - PoliceActionHandler
	\item PoliceActionHandler - ReportRetriever
	\item DataFilter - ReportRetriever
\end{itemize}
\subsubsection{Integration: Application server - RegularUser Client}
Regular user client will be the first one to be integrated with the application server in order for the application to work in its basic functions.
Some client Components will also need an integration with Google maps API to operate correctly.
The specific integrations are:
\begin{itemize}
	\item RegistrationManager - Router
	\item LoginManager - Router
	\item ReportManager - Router
	\item ReportManager - PlateRecognizer
	\item ReportManager - mobile OS camera API
	\item ReportManager - mobile OS GPS API
	\item ReportManager - GoogleMapsAPI
	\item MapViewer - Router
	\item MapViewer - GoogleMapsAPI
\end{itemize}

\subsubsection{Integration: Application server - Policeman Client}
As already said for the regular user client, also policemen client will have components that need an integration with Google maps API and mobile OS APIs. \\ \par
For policemen client, the specific integrations are:
\begin{itemize}
	\item LoginManager - Router	
	\item PoliceActionManager - Router
	\item ReportManager - Router
	\item ReportManager - PlateRecognizer
	\item ReportManager - mobile OS camera API
	\item ReportManager - mobile OS GPS API
	\item ReportManager - GoogleMapsAPI
	\item MapViewer - Google Maps API
\end{itemize}

\subsubsection{Integration: Application Server - MunicipalAuthority Client}
For municipal authorities, the specific integrations are:
\begin{itemize}
	\item SubscriptionManager - Router
	\item LoginManager - Router
	\item RequestManager - Router
	\item RequestManager - Router
	\item RequestManager - MapViewer 
	\item MapViewer - GoogleMapsAPI
	\item CrashDataSender - Router	
	\item MiningService - Router
\end{itemize}

\subsubsection{Integration: load balancer}
The load balancer component will be integrated when the application server development will be complete. The integration of this component may require a remarkable ammount of effort, considering the choice of a third party component. However, we opted for this option considering that the costs of the development of an in-house solution would not be sustainable.

\subsection{Testing Plan}
The system will be tested following a bottom-up strategy.
This means that every single component will be tested individually, followed by the testing of every integration, and finally a general testing of the system.	\par
Individual component testing can take place as soon as the first components' implementation is completed.
This process makes easier to find errors and reduces the possibilities to discover flaws in the following testing phases.
It is important for the component testing to be finished before starting with the implementation phase in order to avoid errors due to the latest changes in the components' algorithms. \par
Individual component testing will be followed by individual Integration testing.
This phase, as the previous one, can be started even if not all the integrations between components are deployed, in order to speed up the process.
Integration will comprehend also connections with external services (such as Google maps API) or device's services (such as mobile OS camera API). These integrations must be tested as well before moving on to the next phase since the most important functionalities are linked to these services. \par
The last step is the system testing, that consists in testing all the functionalities provided by the application altogether, in order to find the errors not resolved by individual testing.